\documentclass[11pt]{article}
\usepackage[T1]{fontenc}
\usepackage{listings}
\usepackage{hyperref}
\usepackage{color}

\definecolor{dkgreen}{rgb}{0,0.6,0}
\definecolor{gray}{rgb}{0.5,0.5,0.5}
\definecolor{mauve}{rgb}{0.58,0,0.82}

\lstset{frame=tb,
  language=c++,
  aboveskip=3mm,
  belowskip=3mm,
  showstringspaces=false,
  columns=flexible,
  basicstyle={\small\ttfamily},
  numbers=none,
  numberstyle=\tiny\color{gray},
  keywordstyle=\color{blue},
  commentstyle=\color{dkgreen},
  stringstyle=\color{mauve},
  breaklines=true,
  breakatwhitespace=true,
  tabsize=3
}
%Template of a bullet list
% \begin{itemize}
%  \item{
%              thie
%
%        }
%\end{itemize}


\setlength{\parindent}{0pt}
\setlength{\parskip}{0pt plus 0.5ex}
%% adjust spacing for all itemize/enumerate

\begin{document}
%TODO: add this in later once everything is complete
%\tableofcontents
%%%%%%%%%%%%%%%%%%%%%%%%%%%%%%%%%%%%Arrays and Hashmaps%%%%%%%%%%%%%%%%%%%%%%%%%%%%%%%
\section{Array and Hashmaps}
\subsection{Group Anagrams}
\href{https://leetcode.com/problems/group-anagrams/submissions/}{link to Question}

The key  idea to solve this question would be that the template of a map could be yet another map.

\begin {lstlisting}
map<map<char,int>,vector<string>> ans;
\end{lstlisting}
\begin{enumerate}
 \item {
       
       For each element in the given string, we would generate a
       \begin {lstlisting}
       map<char,int> obj
       \end{lstlisting}
       where obj would keep count of the number of characters that have appeared in the given string.
       }
       
       \item{
                   Obj would be used as a key to insert the current string in the correct index of the ans map.
             }
 \item {
       Lastly, we would just have to transfer the nested vectors in the ans map to the correct type.
       In this case, it is vector<vector<string>>.
       }
\end{enumerate}
\subsection{Another Question}
$y=mx+c$

\end{document}
